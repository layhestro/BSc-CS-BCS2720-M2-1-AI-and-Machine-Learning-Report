\subsection{Data Discussion}

A significant challenge identified in this study was the imbalance in algorithm usage within the training data. As shown in Figure~\ref{fig:algfreq}, the PPO algorithm appears with a much higher frequency than SAC. Consequently, we opted to split the training data into two distinct subsets—PPO and SAC—rather than training a unified model. While a single model would have been preferable for generalization, the insufficient volume of SAC data made this approach unfeasible. Future iterations of this research should invest more time in data collection to ensure the algorithm distribution is less skewed.

A similar imbalance was observed regarding environment frequencies (see Figure~\ref{fig:envfreqRQ2}). Two primary issues are evident: the overwhelming dominance of the \textit{3DBall} environment and the insufficient representation of several complex environments, such as \textit{FoodCollector}. This imbalance risks biasing the model towards simpler environments while reducing accuracy for underrepresented ones. The solution parallels the algorithm frequency issue: future data collection must prioritize a more balanced distribution of environment types to ensure robust model performance across all tasks.
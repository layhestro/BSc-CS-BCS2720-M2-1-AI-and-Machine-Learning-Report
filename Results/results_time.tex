\subsection{time result}
\begin{figure}[h]
    \centering
    \includegraphics[width=0.5\linewidth]{Results/rq1/r2 comp.png}
    \caption{Comparison of R² scores across all evaluated models for training duration prediction. Higher values indicate better explanatory performance. Error bars represent cross-validation variability.}
    \label{fig:R2 comparison}
\end{figure}

\begin{figure}[h]
    \centering
    \includegraphics[width=0.5\linewidth]{Results/rq1/mae.png}
    \caption{Comparison of mean absolute error (MAE) across all evaluated models. Lower values indicate more accurate predictions. Error bars represent standard deviation across cross-validation folds.}
    \label{fig:mae}
\end{figure}
\begin{figure}[h]
    \centering
    \includegraphics[width=0.5\linewidth]{Results/rq1/comb view.png}
    \caption{Joint comparison of R² and MAE across models. The bottom-right region indicates strong overall performance}
    \label{fig:combview}
\end{figure}
\begin{figure}[h]
    \centering
    \includegraphics[width=0.5\linewidth]{Results/rq1/Picture4.png}
    \caption{Figure A presents the relationship between actual and predicted training durations for the seletcted model}
    \label{fig:gb}
\end{figure}
\begin{figure}
    \centering
    \includegraphics[width=0.5\linewidth]{Results/rq1/Picture5.png}
    \caption{while Figure B shows the corresponding residuals, these figures summarize the predictive performance and error characteristics of the champion model.}
    \label{fig:f5}
\end{figure}
\begin{figure}
    \centering
    \includegraphics[width=0.5\linewidth]{Results/rq1/feature impo.png}
    \caption{Figure C shows the feature importance derived from the Gradient Boosting model for training duration prediction}
    \label{fig:fi}
\end{figure}
\subsection{Training Duration Prediction Results (RQ1)}

\begin{figure}[h]
    \centering
    \includegraphics[width=0.8\linewidth]{Results/rq1/r2 comp.png}
    \caption{Comparison of $R^2$ scores across all evaluated models for training duration prediction. Higher values indicate better explanatory performance. Error bars represent cross-validation variability.}
    \label{fig:R2_comparison}
\end{figure}

\begin{figure}[h]
    \centering
    \includegraphics[width=0.8\linewidth]{Results/rq1/mae.png}
    \caption{Comparison of Mean Absolute Error (MAE) across all evaluated models. Lower values indicate more accurate predictions. Error bars represent standard deviation across cross-validation folds.}
    \label{fig:mae}
\end{figure}

\begin{figure}[h]
    \centering
    \includegraphics[width=0.8\linewidth]{Results/rq1/comb view.png}
    \caption{Joint comparison of $R^2$ and MAE across models. The bottom-right region indicates strong overall performance (high $R^2$, low MAE).}
    \label{fig:combview}
\end{figure}

\begin{figure}[h]
    \centering
    \includegraphics[width=0.8\linewidth]{Results/rq1/Picture4.png}
    \caption{Relationship between actual and predicted training durations for the selected Gradient Boosting model. The red dashed line represents perfect prediction.}
    \label{fig:gb_actual_vs_predicted}
\end{figure}

\begin{figure}[h]
    \centering
    \includegraphics[width=0.8\linewidth]{Results/rq1/Picture5.png}
    \caption{Residual plot showing the difference between actual and predicted values. Random scatter indicates a good fit, while patterns suggest bias.}
    \label{fig:residuals}
\end{figure}

\begin{figure}[h]
    \centering
    \includegraphics[width=0.8\linewidth]{Results/rq1/feature impo.png}
    \caption{Feature importance derived from the Gradient Boosting model for training duration prediction. \texttt{max\_steps} is the dominant predictor.}
    \label{fig:feature_importance}
\end{figure}